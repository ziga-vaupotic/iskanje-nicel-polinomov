\chapter{Uvod}
V računalništvu in informatiki se velikokrat pojavljajo problemi, ki temeljijo na elementarnih matematičnih funkcijah. Najpogosteje se pojavljajo polinomske funkcije. Najpogosteje se pojavljajo na področjih podatkovnih baz (polinomske \textit{hash} tabele), optimizacije podatkov, algoritmov, računalniške grafike, numerične analize, umetne inteligence... Praktično skoraj povsod. Redno se srečujemo tudi s problemi, kjer moramo iskati ničle teh polinomskih funkciji. 

Včasih se je z iskanjem ničel polinomov ukvarjala predvsem numerična analiza \cite{Ridgway2011}. Dandanes numerično analizo interpretiramo kot področje matematike, ki vključuje tudi računalništvo. V zadnjem času se numerična analiza več ne ubada z iskanjem ničel polinomov. S tem se ubada predvsem računalniška algebra, kjer so ničle polinomov veliko bolj pomembne. Načeloma je to področje računalništva (natančneje računalniške matematike), vendar ga veliko ljudi uvršča med področje znanstvenega računanja. Znanstveno računanje je področje podobno numerični analizi \cite{1}, le da je na tem področju računalništvo veliko bolj pomembno kot matematika. Posledično se dokazi ne izpeljujejo analitično vendar empirično. Obstaja pa tudi področje matematična informatika, ki je v resnici samo drugo ime za računalniško matematiko \cite{RačInf}.

Znotraj te naloge si bomo na kratko pogledali nekaj najpomembnejših algoritmov za iskanje ničel ter opredelili njihove lastnosti.

\section{Namen}
Cilj seminarske naloge je opredeliti konvergenco in časovno zahtevnost nekaterih najpogostejših algoritmov za iskanje ničel. Rezultati bodo uporabljeni predvsem za nadaljnjo lastno uporabo. Glavni \textbf{namen} naloge je vzbuditi zanimanje za numerično analizo in uporaba le te v informacijskih tehnologijah. Sicer se redko direktno srečamo s problemom iskanja ničel polinomov v programiranju, a se velikokrat srečamo z optimizacijskimi problemi, ki temeljijo na metodah za iskanje ničel funkciji. Sicer funkcije niso vedno polinomske, a osnovne metode kot so Newton-Raphsovnova, biskecija, ... ostajajo enake. Seveda pa nesemo pozabiti, da lahko funkcije po Taylorjevem izreku aproksimiramo s polinomi. Posledično se velikokrat vrnemo nazaj na metode iskanja ničel polinomov.

\section{Metodologija}
Analiza bo temeljila na pregledu že obstoječe literature. Posvetili se bomo predvsem iskanju ničel polinomov v $\R$ in $\R^n$, seveda pa obstajajo tudi metode za iskanje ničel polinomov v $\mathbb{C}$. V računalništvu imajo kompleksni polinomi vlogo predvsem v teoriji šifriranja in numerični analizi. Vsi algoritmi bodo spisani in implementirani v programskem jeziku Python in bodo dodani kot priloga. Za matematično podlogo bomo uporabili knjižnici \texttt{numpy} in \texttt{scipy}. Za risanje grafov in figur bomo uporabili \texttt{matplotlib}. Algoritmi bodo v nalogi zapisani v obliki pseudo-algoritmeskega zapisa. Izogibali s bomo zapisu nepotrebnih matematičnih izrekov, ter predstavil algoritme tudi grafično, kjer bo to mogoče. Večina programov oz. skript bo časovno analizirana na računalniku s procesorjem \textit{AMD Ryzen 7 2700x}, medtem ko bodo nekateri zahtevnejši algoritmi izvedeni na procesorju \textit{AMD EPYC GENOA 9684X}.

\newpage
\section{Pibližki in napake}
V vseh računalniških algoritmih se pojavljajo napake. Napake se pojavljajo tudi v obliki natančnosti zapisa števil z plavajočo vejico. Po navadi se v obravnavanih algoritmih uporablja podatkovni tip \texttt{double}. Tovrstni podatkovni tip zavzame 8 bajtov.\todo{Dodaj sliko oz. diagram} Prvih 12 bitov predstavlja prolog sestavljen iz predznaka (1 bit) in eksponenta (sledečih 11 bitov). Matisa predstavlja 52 bitov. Natančnost lahko izračunamo z izrazom za plavajočo vejico \cite{Ridgway2011}
\begin{align}
    fl(x) = m \cdot b^{e-1023}
\end{align}
$m$ predstavlja mantiso, torej $0.m_1m_2...m_n$, $b$ predstavlja bazo (ponavadi binarno),  $m_n$ so števke predstavljene v bazne zapisu (posledično so v mejah od 0 do $b-1$), $e$ eksponent, ki je v mejah $L \le e \le U$. Absolutno in relativno napako izračunamo z izrazoma
\[
 E_{abs} = |x - fl(x)| \qquad E_{rel} = \frac{|x - fl(x)|}{x}
\]

V našem primeru imamo na razpolago 11 bitov za zapis eksponenta, torej je zgornja meja $\sum_{i=0}^{10} 2^i = 2047$. To bi bilo sicer res, vendar sta eksponenta $00000000000$ in $11111111111$ \cite{8766229} rezervirana. Tako je največji možen eksponent $11111111110$ oziroma $\sum_{i=1}^{10} 2^i = 2046$. Zanima nas predvsem najmanjši možen eksponent, ki je posledično enak $00000000001$ oz. 1. Če vstavimo navedene podatke v izraz dobimo najmanjšo možno število in posledično največjo možno napako za zapis ničle
\begin{align}
    fl(x) = 0,0\overbrace{...}^{\text{50 ničel}}1\cdot 2^{1-1023} \approx 10^{-308}.
\end{align}
Sicer lahko zapišemo še nižja števila, vendar v tem primeru več nimamo popolne natančnosti decimalk \cite{quadiblocFloatingPointFormats}. Najmanjša možna napaka je enaka najmanjšemu možnem zapisu. Ta napaka velja za vse nadaljnje algoritme.